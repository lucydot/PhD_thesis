\begin{savequote}[8cm]
Energy can neither be created nor destroyed \\
Matter can neither be created nor destroyed \\
Every atom in my body goes back to the Big Bang \\
So rap is no big thang, I was created in the void
  \qauthor{---  Grip Grand, \textit{Conservation of Matter}}
\end{savequote}

\chapter{\label{ch:6-defects}H-centre defects}

\section{Introduction}

%In the previous - we introduced temperature into our model and considered electron-phonon and phonon-phonon interactions. Now we include imperfections in the lattice which are always present at finite temperature.

% many different types of defects, reference earlier, we are interested in "killer defects", carrier capture and NRR as outlined earlier
% interested in minority carrier capture, but roughly equal holes and electronis in perovskite??? so both of interest
% stoneham quote: importance of vibrational properties
% killer defects are associated with lattice distortions. Expect to be localised

% HHPs are defect tolerant - not so many killers.  The `defect tolerance’ of this material is especially surprising given it’s processing method
% - point defects that would be deep have high formation energies- lower formation energies have shallow states- so point defects should not contribute a high density of non-radiative recombination centres
% defects are plentiful: MAPI has low free carrier concentrations due to schottky defect formation which regulates the concentration of charge carriers through ionic compensatino of charged point defects : Self-Regulation Mechanism for Charged Point Defects in Hybrid Halide Perovskites** Aron Walsh,* 
% But the plentiful ones are benign. The literature (http://pubs.acs.org/doi/ipdf/10.1021/jz500370k) says that intrinsic Schottky defects do not lie in the band gap and that Frenkel defects form shallow levels (and so create unintentional doping as reported in experiments). 
% David Mitzi: Voc deficit in MAPI is 0.4mev, in GaAs it is 0.3meV. Impressive to be approachin GaAs after such a short time span.
% or is it? external radiative efficiency in MAPI is 3 orders of magnitude less than GaAs. It is this which differentiate them. MAPI has higher non-radiative recombination. But compared to organics it’s much better. Numbers here: http://www.nature.com/articles/srep06071
%Argument that they are not all benign, as shown by 1\% PL quantum yields.  doi:10.1038/nenergy.2016.149
% open questions around interface and recomgination" JS Park, It has been reported that grain boundaries do not act as charge recombination centres (Edri nanoletters 2014,14,1000-1004)

% point defects in HHPs published. They are a challenge: SoC and Hybrid are expensive. Low symmetry structure due to organic. inequivalent iodine sites. Soft lattice with large distortions: a challenge to model.
% neglect of spin–orbit coupling (SOC) in many previous DFT calculations resulted in incorrect positions of valence and conduction band edges, although their difference, which is the band gap, is in good agreement with the experimental value
% these will be most plentiful. 
% David Cahen: Deep Level Transient Spectrocopy shows that there are shallow levels at 0.28eV, 0.17eV and a deep level at 0.7-0.8eV which has not been confirmed yet.
% Mitzi? Cahne? If we go to the Rose model of 1963 we see that there is a single recombination centre at 0.6-0.8eV, that this is the dominant recombination mechanism under steady state illumination, that recombination is trap mediated.
% Ball et al. nature energy: MAPI has 12 native point defects: VMA, VPb, VI, Interstitials: MA, Pb, I , and anti-site occupations MAPb, MAI, PbMA, PbI, IMA, IPb. "Calculate the formation energy in a host crystal of semiconductor in equilibrium with the pure constituents  which is unlikely given the processing techniques for growing MAPI”. General trend in results is:
% http://pubs.acs.org/doi/abs/10.1021/acs.jpclett.5b00953 "a peak at around T = 191 K is assigned to trap states with activation energies of around 500 meV but with a rather low concentration of 1 × 1021 m–3."
% Pb V is linked ot iodine formation: Pb vacancy leads to I3- trimer, deep state as seen in the PLQE (fillipo de angelis work). This is photoinduced degradation.YY Sun
%can disclude the following from being killer for following reasons...

% iodine interstitial the mobile species - additional incentive to understand as linked to phase segregation and memory effects. There is a link between the ionic and he electronic - unusual as different time scales. To understand, need to understand the carrier capture processes.
% Some defects are introduced under illumination (and then passivated by oxygen): Photoinduced Emissive Trap States in Lead Halide Perovskite Semiconductors 

% for this reason there is a flurry of interest
% In work published prior to...show that forms a trapped hole
% H-center defect castner and kanzig.
% V-centers have been studied in metal halide crystals since the1950s, including F2−impurities in CaF2and LiF, where F−anionsare normally present at specific lattice sites.5,6Related molecularhalide impurities have also been characterized in metal chlorides,bromides, and iodides.7They are unusual point defects as theyinvolvenomissing or extra atoms. A hole is introduced into apristine lattice, and the charge induces the formation of a bondbetween two nearest-neighbor halide ions, producing an open-shell dihalide species, 2X−+h+→X2−. In contrast to the V-center, the H-center is where hole-capture and dimer formationinvolves excess halide (for example, an interstitial) interactingwith a lattice site, X−+Xi−+h+→X2−. It has been shown that H-centers can be thermally converted into V-centers, and the twodefects have similar but distinguishable optical and electronicsignatures.5
% schematic made by Aron.
% since then there has been following literature
% we give carrier capture rate
% we have predicted H-center defect, and flurry of interest. No carrier capture rate.



\section{Methods}

\subsection{Defect formation energies}
% ionic relaxation:PREC= NORMA LREAL = AutoEDIFF = 1.0e-05EDIFFG = -0.01ENCUT = 550ISMEAR = 0SIGMA = 0.1ISPIN = 2 (spin-polarised)
% The interstitial is placed in a 192-atom supercell, which is built from the 2\sqrt2x2\sqrt2x2 expansion of the 12-atom cubic cell, using the transformation matrix [2 -2 0 // 2 2 0 // 0 0 2]
%Ionic relaxation at HSE06 confirms that there is very little distortion from PBEsol relaxed structure.
% Perfect structure confirmed stable from a phonon calculation.

% Using the HSE06 hybrid functional with spin-orbit coupling, and a 2x2x2 sampling of the electronic brillouin zone (gamma centred).PREC= NORMALREAL = AutoLCHARG = .TRUE.EDIFF = 1.0e-06ENCUT = 550SIGMA = 0.1. bandgap is correct with correct alpha.

% Using Daint. Importance of ROPT. Importance of alpha parameter

% convergence studies with number of atoms



\subsection{Lattice dynamics}

\subsection{Carrier capture rates}

% @sunghyun will have to correct me if I'm wrong but my understanding is that the waveunctions generated by vasp are not orthogonal so its not valid to calculate the overlap using the dot product. Pawpyseed can be used to orthogonalise the wavefunctions and then dot products can be done
% Yes, a WAVECAR file contains only the plane-wave coefficients for the smooth pseudo wave function and we also need the core part to do anything with the all-electron wave function.
\section{Results} \label{ch:6-results}

\subsection{Defect geometry}
% consider in all charge states due to where the fermi nergy is.
% schematic of lowest energy structures found: -/neutral/positive.
% description of each - in-plne vs out of plane
% how this compares to literature
%  bonding lengths - is this closer to optimal? lattice distorts so can get close?
% include significance of large distortion: in plane to out of plane
% include displacement vectors
% talk about breaking symmetry. Why we chose cubic: not decided already how broken.
% need to manually break symmetry - testing for lower energy.
% different depending on starting point: pockets of minima

\subsection{Charge localisation}
%plot spin density
% including TDDFT
% Following the“molecule in crystal”approach7with dielectricembedding, we computed the optically excited states of the Cl2−,Br2−, and I2−defect centers as a function of bond lengths usingtime-dependent DFT (seeFigure 2) including relativistic effect
% However, certain point defects can act as color centers,with charge-conserving optically excited states. For example, thecolor (F) centers consisting of an electron trapped at a vacantlattice site in alkali halides, or self-trapped holes in heteropolarcrystals that are termed V-centers.
%Previous section look at large polaron, in this one a small polaron is considered.

\subsection{Defect formation energies}

% formula
% defect corections very small: do sxdefectalign output in appendix
% using previously published dielectric for ionic?
% tilting correction results: 
    %We expect the tilting correction to scale linearly with supercell size
    %See work published in Ruoxi's thesis
    %270/37 = 7.5 ~ 8 which is to be expected as supercell is 8* bigger
    %We expect reduction of 2*37meV for 192 supercell = 0.074eV
     % Schematic of what is happening Although small, important as exponential term.
% negative U behaviour

\subsection{Vibrational properties}
% including IPR

\subsection{Configuration Coordinate diagram}

% Need to make CC diagram. 
% use linear interpolation
% interpert CC
% CC diagram with schrodinger solution superimposed.
% problem with the adiabatic approach and falling to the local minima



\subsection{carrier capture rate}

% how carrier capture is calculated - alkauskas.
% importance of suitable starting state and state to capture into
% then the eigenvalues with distortion and overlap/gradient of overalpt....

% guidance for synthesis?

\textbf{Summary}
% Future work around the interaction between electronic and ionic states.
%  Themolecular iodide defects discussed here are not necessarily staticcarrier traps, because the halides that form the corner-sharingperovskite framework support reasonable rates of ion transport.Room-temperature diffusion is expected for the V-center, whichcould proceed in a pathway akin to vacancy-mediated diffusion.
% Transport of the H-center would require an interstitialcymechanism, which has also been predicted to be low energy inhalide perovskites.12The slow motion of trapped holes wouldadd a further layer to the complexity of the temporal response ofperovskite solar cells to light soaking and bias voltages (alsoevident in quantum dot photovoltaics13). Given the polyanionnature of iodine (e.g., charged I2up to I16complexes) and theflexibility of the perovskite structure, the formation of largercharged molecular aggregates is also possible for the iodideperovskites. Such polyanion inclusions in a crystal would beredox active and could facilitate additional electron or holetrapping, which is a fertile line of research for future studie
% interstitcalicy - sam stranks paper

\textbf{Data access statement}
% I need to publish my structures!
% displacement vectors
% carrier capture code

\textbf{Acknowledgements}

Calculations were performed on the SiSu supercomputer at the IT Center for Science (CSC), Finland, via the Partnership for Advanced Computing in Europe (PRACE) project no. 13DECI0317/IsoSwitch.
% Need to include TDDFT at start thesis