Hybrid halide perovskites are being developed for use as an absorber material in solar cells, alongside other optoelectronic applications such as the light-emitting diode or laser. Research interest in this material family has grown quickly over the decade, as photovoltaic efficiencies have increased from x\% in 2010 to the current record of y\%.  In addition, the synthesis procedure is a low-temperature solution-deposition method which, when commercialised, may allow for a reduction in solar module production prices. 

Materials theory and simulation has struggled to keep up with this rapid progress as many of the physical processes which determine solar cell performance are related to defects (e.g.\ carrier capture and recombination) and temperature (e.g.\ degradation and ion migration), which are challenging to model from first-principles. Firstly, a typical Density Functional Theory (DFT) calculation for a crystalline material assumes that the material is perfectly periodic, with no point or extended defects. Secondly, DFT is used to model ground-state properties; to account for temperature effects (e.g.\ atomic vibrations) it is necessary to combine DFT with other methods (e.g.\ lattice dynamics). The aim of this PhD project is to move away from the idealised picture of a perfect material at absolute zero and towards a more realistic picture, where the defects and distortions of hybrid halide perovskites are considered.