Hybrid halide perovskites are being developed for use as an absorber material in solar cells, alongside other optoelectronic applications such as a light-emitting diode emitter or lasing material. Research interest in this material family has grown quickly over the decade, as photovoltaic efficiencies have increased from 10.9\% in 2012 to the current record of 24.2\%.  In addition, the synthesis procedure is a low-temperature solution-deposition method which, when commercialised, may lead to a reduction in solar module production prices. 

Materials theory and simulation has struggled to keep up with the rapid experimental progress as many of the physical processes that determine solar cell performance are related to defects (e.g.\ carrier capture and recombination) and temperature (e.g.\ degradation and ion migration), which are challenging to model from first-principles. Density Functional Theory (DFT) is used to model ground-state properties only, and a typical DFT calculation for a crystalline material assumes that the material is perfectly periodic, with no point or extended defects. To account for temperature effects (e.g.\ atomic vibrations) or defects it is necessary to combine DFT with other methods (e.g.\ lattice dynamics or defect corrections). The aim of this PhD project is to move away from the idealised picture of a perfect material at absolute zero and towards a more realistic picture, where the defects and distortions of hybrid halide perovskites are considered.

% 10.9 from snaith paper:https://science.sciencemag.org/content/338/6107/643
% 24.2 from Solar cell efficiency tables (version 54)