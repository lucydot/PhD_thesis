I declare that this thesis and the work presented in it are my own and has been generated by me as the result of my own original research. This work has included: identifying research questions, preparing input files for calculations, submitting and monitoring calculations, writing scripts and software packages for analysis, interpreting the results and writing research papers. Where work is not my own references are given. In addition, I list below the instances where work has been done in conjunction with others.
\vspace{\frontmatterbaselineskip}

\textbf{Theory and simulation of hybrid halide perovskites } 

%The text in this chapter is largely reproduced from a published paper.\autocite{Whalley2017} The lead author is myself and co-authors are Jarvist M. Frost, Young-Kwang Jung and Aron Walsh.
The central idea of this chapter (to review the theory and simulation of hybrid halide perovskites) was provided by Aron Walsh. The contents of the review were a product of discussions between myself, Aron Walsh and Jarvist Frost. Aron Walsh prepared figures 2.1, 2.2--2.4, and Young-Kwang Jung prepared figure 2.3. This chapter is a literature review; the primary research underpinning this chapter was performed by Jarvist M. Frost (molecular dynamic and Monte Carlo investigations), Federico Brivio (crystal and electronic structure), Jonathan M. Skelton (lattice dynamics and vibrational spectroscopy), and myself (band-gap deformations).

\vspace{\frontmatterbaselineskip}

\textbf{Electronic band non-parabolicity}

%The text in this chapter is largely reproduced from a published paper.\autocite{Whalley2019} The lead author is myself and co-authors are Jarvist M. Frost, Benjamin J. Morgan and Aron Walsh. 
The initial research direction (to investigate, across a range of photovoltaic materials, the sensitivity of DFT calculated effective mass to fitting parameters) was provided by Aron Walsh and Benjamin Morgan. Jarvist Frost suggested weighting the fit to a Fermi-Dirac distribution. 

\vspace{\frontmatterbaselineskip}

\textbf{Electron-phonon and phonon-phonon coupling}

%The text in this chapter combines results from two published papers.\autocite{Whalley2016,Whalley2017a} The lead authors are myself\autocite{Whalley2016} and Jarvist M. Frost,\autocite{Whalley2017a} and the co-authors are Jonathan M. Skelton and Aron Walsh. 
Jarvist Frost suggested using a classical heat diffusion model for hot carrier cooling and calculated the temperature dependent bandgap shifts. Jonathan M. Skelton provided scripts to implement the frozen phonon method. Aron Walsh prepared figure 5.1.

\vspace{\frontmatterbaselineskip}

\textbf{H-centre defects}

% ADJUST BELOW
%The text in this chapter includes work from a published paper.\autocite{Whalley2017b} The lead author is myself and co-author is Aron Walsh. 
The central idea of this chapter (to investigate hole capture at an iodine interstitial in \ce{CH3NH3PbI3}) was provided by Aron Walsh. Sunghyun Kim and Samantha Hood provided early (pre-publication) access to the software package \textsc{CarrierCapture.jl}, which I used to analyse the data presented in Section \ref{ch:6-results}. Aron Walsh prepared figure X.