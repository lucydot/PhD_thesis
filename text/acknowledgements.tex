Leaving an enjoyable job and returning to study is not an easy decision, but starting a PhD in the Walsh group was one of the best I have made. A huge thanks must be given to Aron Walsh for creating and maintaining a supportive, friendly and productive working environment. I have gained a \textit{huge amount} from working with Aron and other members of the group. Although I have learnt something from every member I will not list everyone here, but a special thanks must go to Jarvist Frost who led book group and helped clarify my sometimes muddled understanding of solid state physics. Another special thanks to the women I have worked with; we are still too few and far between in the physical sciences, but what we lack in quantity we make up for in quality! Thanks also to Benjamin Morgan, my second supervisor, who guided me through my first research project and has provided extensive feedback on my scientific writing. 

During the third year of my PhD I became involved with the Research Software Engineering community at Imperial. I would like to thank Katerina Michalickova for organising the Software Carpentry workshops - it is no easy job co-ordinating a rabble of volunteers from across Imperial, but teaching software carpentry has greatly increased my confidence around programming. Thanks also to fellow members of the Imperial RSE community committee, especially Jeremy Cohen who is mentoring me as part of my Software Sustainability Institute fellowship.

The vast majority of the computational analysis in this thesis depends to some degree on the open-source community. Thousands of person-hours, often I suspect squeezed in at weekends or evenings, has produced well-tested, well-documented, easily accessible code. Without the magical Python `import` statement and the generosity of strangers I would probably still be writing an integration routine for my first year project. Thanks also to the admin and technical staff who keep the supercomputers Archer, Thomas and Piz-Daint running. I rarely think about the infrastructure behind my calculations, and this is exactly why I need to say thanks.

Funding for my PhD came from the EPSRC, via the Centre for Doctoral Training in New and Sustainable Photovoltaics. It seems that running a CDT requires staff to go above and beyond - thanks to Ken Durose, Alison Walker, Rob Treharne and Asim Mumtaz for doing just this.

The biggest thanks must go to my family and friends. I met Richard (in a \textit{romantic} way) towards the start of my PhD. It is now four years on, and we have bought a house, got married and created a very small human being (currently still residing in my tummy). I wouldn't have expected that all this was compatible with doing a PhD, but being with Richard has made it seem like the most natural thing in the world. He has been with me, and supported me, throughout. It is through Richard's parents, Andrew and Elspeth, that I was introduced to John and Karen. By gifting me use of a bedroom twice a week for three years, John and Karen have enabled me to live in Birmingham and work in London. I don't think it would have been possible for me and Richard to `settle down' otherwise and for this I am very grateful.

I'm lucky to have close, reliable and hilarious friends who are always suggesting things to take my mind of work. Thanks especially to Hetta and Blanche for Lipsync and our other little ``projects'' and Caz for her loyalty and warmth. Ruby, my sister, is equally hilarious and understands me (and sees through me!) in a way that no-one else can. The final thanks is reserved for my parents John and Alison, whose love and support instilled in me a self-confidence that I think is so important when doing science. All of my achievements are theirs, as none of this would have been possible without their investment into me.