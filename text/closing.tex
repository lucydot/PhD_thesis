

Many oppotunities ahead as we pick apart.....(from end of review)
include outstanding research questions from late stage review

For my own work:
not touched on is toxicity and drive to replace lead.The lead is important for transport properties. The lead is in a 2+ charge state. This means it has the chemical structure 6s2 6p0. This means that the dispersive s-orbital is active in the valence band. So when charge electrons are excited to the conduction band the holes are in a dispersive valence band and so can provide ambi-polar transport. The iodine provides dispersive conduction band edge (rom real-space structure can see there must be hybridization as there is no Pb overlap). Pb has the edge over Sn based compounds as the s electrons in Pb are deeper down and so harder to remove. In Sn based perovskkites the Sn2 wants to form Sn4 which kills the transport (dispersion) and leads to breakdown of the material (formation of different phases) – the Sn wants to oxidise.
 concentrated on protoypical to understand fundamentals
 
 Also ferroelectricity and Evidence for ferroelectric superhighways (predicted in atomistic simulations…): Tetragonal MAPI is ferroelctric from david cahen group. and starrynight.
 
 Future of PV:
 - being combined with batteries for continuous power, is this the limiting?The other big thing is storage networks. This and BIPV listed as things for large scal PV deployment in Raugil nad Frankl 2009
- BIPV: This is what Dyesol company want to do. They are not going to compete with silicon on the arrays of solar farms. Going to look into market where silicon has not penetrated: where it cannot because of need to be solid single crystal. Aim is functionalisation of buildings with PV. Printing on steel.
 
 It seems that there are technological solutions, but not the political will: what should the role of scientist be?
 
 Modelling - baby in the bathtub. In a complex system with so much interconnected
 
 Analysis software has been made open, testing, documentation - reproducibility in science / reusable
 
 If our motivation is environment, need to think of as whole - battery / timescales
 
 I motivation is for the understanding itself, there is a lot left to understand in hybrid perovskite materials.
 
 Future - ML - https://journals.aps.org/prl/pdf/10.1103/PhysRevLett.122.225701

For the wider field:

Need to move to offline
Reproducibility