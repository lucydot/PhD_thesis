The subject of this PhD thesis has been to investigate distortions (in the form of nonparabolic electronic bandstructures and anharmonic potential energy surfaces) and defects (in the form of the iodine interstitial) in hybrid halide perovskites. There are three main components in this work:

First I assessed the impact of band nonparabolicity on transport and optical properties at high carrier concentration. \textbf{I have shown that effective masses of PV materials are not constants and that this leads to a reduction in carrier mobility at high carrier concentration}. I have written and published a software package (Appendix \ref{app:1-effmass}) so that researchers can apply my methodology to their system of interest; application to materials that operate at high carrier concentrations, such as transparent conducting oxides or thermoelectric materials, could be particularly interesting.

I then considered electron-phonon and phonon-phonon coupling in the hybrid halide perovskite \ce{CH3NH3PbI3}. \textbf{My results show that there is strong coupling between the electronic states at band edge and the anharmonic phonon modes associated with tilting of the inorganic octahedra}. Using parameters derived from the Fr\"{o}lich polaron model, \textbf{I have found that the cooling of photoexcited, above bandgap carriers is limited by the ultralow thermal conductivity of the perovskite lattice and that cooling to equilibrium happens over 100s of picoseconds}. This compares well to the experimentally observed time scale of slow carrier cooling.

Finally, I studied the iodine interstitial defect. \textbf{I predict that holes are self-trapped at this defect to form a H-centre defect ($\mathbf{I_2^-}$), and that the rate of hole trapping is X}. My research prompts a number of questions for future work:
\begin{itemize}
    \item Does hole capture lead to device degradation? Hybrid halide perovskites have an ultra-low thermal conductivity\autocite{Whalley2016} which, when combined with fast carrier trapping (where electronic energy is converted to lattice energy), could lead to a build up of highly localised heat. This local heating could accelerate device degradation; experimental reports show that decomposition of MAPbX (X = I, Br) can be triggered by a raman laser.\autocite{Ledinski2015}
    \item Do stochiometric iodine pairs form V-centres? It is energetically favourable to form V-centres in metal halides such as KCl,\autocite{Castner1957} but formation of this defect in the the metal halide perovskites has not been considered. If favourable, the V-centre will form in high concentrations as it is a stoichiometric defect (requires no excess or missing atomic species). The long diffusion length but mediocre mobility of charge carriers in hybrid halide perovskites\autocite{Brenner2015} supports the existence of the V-centre, although this could also be accounted for by other types of polaron formation. 
    \item Is defect energy depth a good measure of wavefunction localisation? In the literature a ``deep'' defect state (with a charge transition level towards the centre of the bandgap) is often correlated with highly localised charge. This correlation between the defect depth and localisation is often assumed (with the terms sometimes used interchangeably), but to my knowledge there is no systematic study of this relationship. A potential study could consider e.g. anion vacancies across CdTe, GaAs and MAPI.
\end{itemize}

During my PhD studies I have made key datasets and analysis scripts openly available through the GitHub platform. My reasons for doing so are three-fold: for increased reproducibility; to reduce duplication of effort and so make efficient use of public money and researcher time; and to promote my work. Reproducibility is surprisingly difficult to achieve -- I have found that publishing software is not enough, and that software documentation and testing is needed to ensure that others can independently reproduce my results. It has taken time to develop the skills needed to write, document, test and publish code, but I consider it time well spent.

In the first chapter I motivated this work with reference to global warming. However it is clear that efficient photovoltaic materials alone will not slow global warming, and that the larger picture must be considered. For example, if PV is to provide a large proportion of our energy supply, it will need to be combined with large scale energy storage networks. If we are serious about climate change as the motivating factor, it may be that we have to look beyond science, as some claim that it is the political will, rather than technological solutions, that are lacking. 

If, however, we believe that scientific understanding is an end in itself, there is plenty of motivation for further research into hybrid perovskite materials. In addition to the open questions listed above, there is an important question relating to the metal B-site species -- is it possible to replace lead with a non-toxic element? Efforts so far have failed as the chemical structure of lead is such that the dispersive s-orbital is active in the valence band, leading to excellent transport properties that cannot be reproduced with tin on the B-site, for example.
There are many opportunities ahead as we pick apart the relationship between organic and inorganic components, electronic and ionic states, as well as order and disorder in this complex family of materials.


%not touched on is toxicity and drive to replace lead.The lead is important for transport properties. The lead is in a 2+ charge state. This means it has the chemical structure 6s2 6p0. This means that the dispersive s-orbital is active in the valence band. So when charge electrons are exautocited to the conduction band the holes are in a dispersive valence band and so can provide ambi-polar transport. The iodine provides dispersive conduction band edge (rom real-space structure can see there must be hybridization as there is no Pb overlap). Pb has the edge over Sn based compounds as the s electrons in Pb are deeper down and so harder to remove. In Sn based perovskkites the Sn2 wants to form Sn4 which kills the transport (dispersion) and leads to breakdown of the material (formation of different phases) – the Sn wants to oxidise.

